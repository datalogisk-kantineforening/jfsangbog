\section{\huge{Bonusopgaver}}

\subsection{Opgave 0}

...

\newpage

...

\newpage

...

\newpage

...

\newpage

...

\subsection{Den nemme opgave}

% The Collatz conjecture

Her er en funktion for et vilkårligt tal $n$:
\begin{align*}
f(n) = \begin{cases}
n/2 &\text{hvis }n\text{ er lige}\\
3n + 1 &\text{hvis }n\text{ er ulige}\\
\end{cases}
\end{align*}
For eksempel er $f(10) = 5$ og $f(7) = 22$.

Her er en talfølge -- baseret på funktionen $f$ -- der begynder med et
vilkårligt tal $n$:
\begin{align*}
a_i = \begin{cases}
n &\text{når }i = 0\\
f(a_{i - 1}) &\text{når }i > 0\\
\end{cases}
\end{align*}
Hvis vi for eksempel sætter $n = 12$, får vi talfølgen
$a_0 = 12, a_1 = 6, a_2 = 3, a_3 = 10, a_4 = 5, a_5 = 16, a_6 = 8, a_7 = 4, a_8
= 2, a_9 = 1$.  Vi stopper med at vise en talfølge når den når tallet $1$, for
$f(f(f(1))) = 1$ (vis dette), så den vil blot gentage sig.

\textbf{Din opgave:} Bevis eller modbevis følgende formodning:
\begin{quote}
Ligemeget hvilket $n$ der vælges, vil talfølgen altid nå tallet $1$.
\end{quote}

\textbf{\emph{NB: Der udloddes en flaske snaps til den første som kommer op i
baren med en korrekt besvarelse af denne opgave!}}
